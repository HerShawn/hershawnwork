% multiple1902 <multiple1902@gmail.com>
% intro.tex
% Copyright 2011~2012, multiple1902 (Weisi Dai)
% https://code.google.com/p/xjtuthesis/
%
% It is strongly recommended that you read documentations located at
%   http://code.google.com/p/xjtuthesis/wiki/Landing?tm=6
% in advance of your compilation if you have not read them before.
%
% This work may be distributed and/or modified under the
% conditions of the LaTeX Project Public License, either version 1.3
% of this license or (at your option) any later version.
% The latest version of this license is in
%   http://www.latex-project.org/lppl.txt
% and version 1.3 or later is part of all distributions of LaTeX
% version 2005/12/01 or later.
%
% This work has the LPPL maintenance status `maintained'。
%
% The Current Maintainer of this work is Weisi Dai。
%

\chapter{基于边缘骨架切割的文字检测方法}
\echapter{Saliency based Defect Detection Method}

    \section{问题的提出}
    \esection{Outline}

    \section{方法原理与步骤}
    \esection{Saliency Map Detection}

    \section{文字骨架的提取与切割}
    \esection{Edge Detection}

        \subsection{文字骨架的生成}
        \esubsection{Improved Saliency based Detection}

        \subsection{文字骨架粘连点的检测和切割}
        \esubsection{Improved Saliency based Detection}

    \section{非文字骨架的滤除}
    \esection{Morphology Operation}
    
        \subsection{形态学滤除}
        \esubsection{Improved Saliency based Detection}

        \subsection{基于卷积神经网络的滤除}
        \esubsection{Improved Saliency based Detection}

    \section{文本行生成}
    \esection{Morphology Operation}

        \subsection{最稳定极值区域的提取}
        \esubsection{Improved Saliency based Detection}

        \subsection{文本行的局部迭代优化}
        \esubsection{Improved Saliency based Detection}

    \section{实验和结果分析}
    \esection{Experimental Results}
    
        \subsection{数据集与评价方法}
        \esubsection{Data-set and Evaluation Protocol}

        \subsection{实验结果与分析}
        \esubsection{Experimental Results and Analysis}

    \section{本章小结}
    \esection{Brief Summary}


