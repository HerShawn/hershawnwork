% multiple1902 <multiple1902@gmail.com>
% intro.tex
% Copyright 2011~2012, multiple1902 (Weisi Dai)
% https://code.google.com/p/xjtuthesis/
%
% It is strongly recommended that you read documentations located at
%   http://code.google.com/p/xjtuthesis/wiki/Landing?tm=6
% in advance of your compilation if you have not read them before.
%
% This work may be distributed and/or modified under the
% conditions of the LaTeX Project Public License, either version 1.3
% of this license or (at your option) any later version.
% The latest version of this license is in
%   http://www.latex-project.org/lppl.txt
% and version 1.3 or later is part of all distributions of LaTeX
% version 2005/12/01 or later.
%
% This work has the LPPL maintenance status `maintained'。
%
% The Current Maintainer of this work is Weisi Dai。
%

\chapter{基于边缘骨架切割的文字检测方法}
\echapter{Text detection based on Skeleton-cut detector}

    \section{问题的提出}
    \esection{Questions Posed}

    \section{方法原理与步骤}
    \esection{Principle and Summary of The Method}

    基于边缘骨架切割的文字检测子的具体流程如图\ref{fig.c31_overview}所示。首先对于输入的场景文字图片,利用结构化边缘检测方法\cite{Dollar2015Fast}得到输入图片的结构化边缘响应图。然后基于一系列像素强度值阈值,对边缘响应图进行分割,得到相应的二值化边缘图。

    \begin{figure*}[!h]
    \centering
    \includegraphics[width=\textwidth]{./figures/c31_overview_1.jpg}
    \begin{minipage}[t]{0.32\linewidth}
    \centerline{ \small (a)输入的场景文字图像}
    \end{minipage}
    \begin{minipage}[t]{0.32\linewidth}
    \centerline{ \small (b)场景文字图的边缘响应}
    \end{minipage}
    \begin{minipage}[t]{0.32\linewidth}
    \centerline{ \small (c)场景文字图的边缘骨架}
    \end{minipage}
    \includegraphics[width=\textwidth]{./figures/c31_overview_2.jpg}
    \begin{minipage}[t]{0.32\linewidth}
    \centerline{ \small (d)边缘骨架切割}
    \end{minipage}
    \begin{minipage}[t]{0.32\linewidth}
    \centerline{ \small (e)非文字边缘骨架图的滤除}
    \end{minipage}
    \begin{minipage}[t]{0.32\linewidth}
    \centerline{ \small (f)文字粗略定位结果}
    \end{minipage}
    \caption{基于边缘骨架切割的文字检测方法的流程图}
    \label{fig.c31_overview}
    \end{figure*}

    \section{文字骨架的提取与切割}
    \esection{Text skeleton's extraction and cutting}

        \subsection{文字骨架的生成}
        \esubsection{Text skeleton's construction}

        \begin{figure*}[!h]
        \centering
        \includegraphics[width=\textwidth]{./figures/c32_skeleton.jpg}
        \begin{minipage}[t]{0.24\linewidth}
        \centerline{\small (a)原图}
        \end{minipage}
        \begin{minipage}[t]{0.24\linewidth}
        \centerline{\small (b)边缘响应}
        \end{minipage}
        \begin{minipage}[t]{0.24\linewidth}
        \centerline{\small(c)边缘二值图}
        \end{minipage}
        \begin{minipage}[t]{0.24\linewidth}
        \centerline{\small (d)边缘骨架图}
        \end{minipage}
        \caption{文字边缘骨架生成的流程图}
        \label{fig.c32_skeleton}
        \end{figure*}

        \subsection{文字骨架粘连点的检测和切割}
        \esubsection{Text adhesion-junctions' detection and cutting}

    \section{非文字骨架的滤除}
    \esection{Non-text skeleton's filtering}

        \subsection{形态学滤除}
        \esubsection{Morphological filtering}

        \subsection{基于卷积神经网络的滤除}
        \esubsection{convolutional neural network-based filtering}

    \section{文本行生成}
    \esection{Text line's construction}

        \subsection{最稳定极值区域的提取}
        \esubsection{Extraction of the most stable extremum region}

        \subsection{文本行的局部迭代优化}
        \esubsection{Text line's iteratively local refinement}

    \section{实验和结果分析}
    \esection{Experimental Results}

        \subsection{实验数据集与评价标准}
        \esubsection{Data-set and Evaluation Protocol}

        \subsection{实验结果与分析}
        \esubsection{Experimental Results and Analysis}

    \section{本章小结}
    \esection{Brief Summary}


