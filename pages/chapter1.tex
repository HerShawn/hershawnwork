% multiple1902 <multiple1902@gmail.com>
% intro.tex
% Copyright 2011~2012, multiple1902 (Weisi Dai)
% https://code.google.com/p/xjtuthesis/
%
% It is strongly recommended that you read documentations located at
%   http://code.google.com/p/xjtuthesis/wiki/Landing?tm=6
% in advance of your compilation if you have not read them before.
%
% This work may be distributed and/or modified under the
% conditions of the LaTeX Project Public License, either version 1.3
% of this license or (at your option) any later version.
% The latest version of this license is in
%   http://www.latex-project.org/lppl.txt
% and version 1.3 or later is part of all distributions of LaTeX
% version 2005/12/01 or later.
%
% This work has the LPPL maintenance status `maintained'.
%
% The Current Maintainer of this work is Weisi Dai.
%

\chapter{绪论}
\echapter{Introduction}

    \section{背景}
    \esection{Background}
    近年来,连铸坯表面缺陷检测在钢铁行业中受到越来越多的关注,连铸坯热送热装以及连铸技术\cite{Association2012World}取得了很大进步,然而要实现连铸坯热送热装首先需要研究无缺陷的连铸坯制造技术。

    作为一项系统性的新工艺,高温连铸坯热送热装及直接轧制技术能够直接影响到炼钢流程的一体化程度,实现这一技术能够降低生产成本、节省生产能源、缩短生产周期以及提高产品质量。然而,要实现这一工艺,需要研究热态连铸坯在线缺陷检测技术,实现这一技术可避免带缺陷的连铸坯毫无意义的继续深加工,而无缺陷的连铸坯能够直接送入下一步热轧工艺。

    目前,国内大部分钢铁企业采用的主要检测手段还是依靠人工目测的方法判断连铸坯表面质量,对连铸坯表面质量的修复主要通过将连铸坯冷却至低温后,人工目测出缺陷然后采用人工火焰清理和抽样酸洗方法来完成,这种检测方法的准确率倾向于人工经验,无法实现实时在线检测,而且生产率较低,不能满足连铸坯缺陷的在线剔除和热送热装的要求。

    由于人工检测较为依赖人工经验,并且需要先将连铸坯冷却,无法实现实时的在线检测,所以实现对高温连铸坯的实时表面缺陷检测系统,可以有效的提高连铸坯热送热装以及直接轧制率,进而能够节约能源、降低生产成本。研究连铸坯表面缺陷检测算法的目标就是:使用深度信息实时检测出连铸坯表面的缺陷,连铸坯表面几乎全部缺陷是深度相关的,使用RGB信息很难检测出来。检测出缺陷后,根据检测出的缺陷位置就能够定位缺陷,然后结合修磨机能够实现连铸坯表面缺陷定点清除,这样就可以减少人工检测的情况,使得合格的高温连铸坯能够直接输送到热轧工序,提高热送率,降低能源成本,提高生产率。

    \section{国内外研究现状}
    \esection{Related Work}
    自连铸机诞生以来,高温状态下进行连铸坯表面的实时缺陷检测,一直是各个工业国家比较重视并竞相发展的一项技术。自研究工作开始以来,不少研究工作者就利用基于光学、超声波和涡流等多种物理探伤手段对连铸坯进行表面缺陷检测。这种检测方法是利用光、热、声、电、磁等物理反应在材料表面或者内部的异常区域反馈信号的变化,根据这一特点人们通过各种手段开展了对连铸坯表面无损探伤技术的研究。

    关于热态连铸坯表面缺陷检测,在传统无损检测方面,法国钢铁研究院和索丽梅公司首次运用电涡流对连铸半成品进行探伤技术的研究,冶金部钢铁研究总院的贾慧明\cite{贾慧明19941100} 在实验室内通过对特定钢种制造人工伤的方法完成了1100摄氏温度以上高温铸坯涡流缺陷检测实验,西安交通大学的张庆军\cite{张庆军2005涡流检测技术在钢铁工业中的新应用}对涡流技术在钢铁工业中的总体应用做了相关的介绍。

    在光学检测方面,重庆大学的欧阳奇等\cite{欧阳奇2007高温连铸坯表面缺陷的机器视觉无损检测}展开了高温连铸坯表面缺陷机器视觉无损检测的理论和实验研究,美国芝加哥的Parsytee缺陷检测公司研究了面阵CCD成像检测系统在钢厂的应用,并利用多个面阵CCD视场相互交叠采集图像的方法扩大了钢板宽度检测范围,北京航空航天大学的王志成等\cite{王志成2009钢板表面缺陷检测系统的设计与实现}基于图像的方法开始了对钢板缺陷识别系统的设计与实验工作,使用模式匹配的方法对钢板进行了缺陷检测。

    在深度信息检测方面,北京科技大学的徐科等\cite{徐科2008基于线型激光的热轧带钢表面在线检测系统}利用线性激光器来检测钢板表面缺陷的深度信息,这种方法比较简单,主要是利用激光器、CCD相机和钢板表面的空间关系来测量缺陷的深度信息。Landstrom 等\cite{Landstrom2012Morphology} 利用多角度形态学操作检测裂纹缺陷,并使用逻辑回归利用缺陷的统计特征进行分类。zhao 等\cite{Zhao2014Defect} 构建了一个板坯表面缺陷检测系统,他们先获得可能存在缺陷的局部连通区域,然后利用深度信息提取缺陷和背景种子点,接着使用模糊连接算法确定缺陷轮廓。

    总体来讲,使用电涡流检测技术,容易受到电涡流探测器本身提离效应的约束,而且对激励信号的要求也比较高,因此目前国内外在高温连铸坯缺陷检测方面的电涡流检测方法尚未有突破进展。采用传统光学方法进行检测时,由于受到连铸坯表面氧化铁皮和水膜等物质的信息干扰,利用常规的图像处理方法很难从CCD图像中将二者与缺陷特征区分开,容易造成错误的识别。而利用深度信息能够很好的降低氧化铁皮和水膜等物质的误检率。

    深度图像相对于传统的RGB图像而言,能够表示三维空间的信息,传统的RGB图像将场景从三维转换到了二维,于是深度距离信息也就丢失了。目前连铸坯表面缺陷主要分为:夹渣、气泡、机清沟、裂纹还有划伤等等,这些缺陷大部分都具有深度信息,而且利用深度信息也能够很好的区分真正的缺陷和氧化铁皮等伪缺陷。目前而言利用深度信息的三维缺陷检测是发现热态连铸坯表面缺陷的最有效手段。


    \section{本文研究内容}
    \esection{Research Contents}
    本文提出了两种利用深度信息检测连铸坯表面缺陷的算法,并采用人工合成的方法生成了一个深度图像数据库。这两个缺陷检测算法在深度图像数据库上的实验结果显示它们都能够有效地检测出带有深度信息的缺陷。本文的主要研究工作如下:

    1)提出了一种基于显著性图的缺陷检测算法,主要思路是首先是用改进的显著性检测来获取图像的显著性图并去除图像中的黑斑,然后使用canny边缘检测提取边缘图像,最终使用形态学操作对边缘图像进行处理获得最终的缺陷区域;

    2)提出了一种基于基准面的深度图像缺陷检测算法,主要思路是根据输入的深度图像拟合出其基准面,再根据拟合的基准面以及原始深度图像获得缺陷种子点,最后在原始深度图像上使用基于种子点的图像分割算法获取最终的缺陷轮廓;

    3)提出了一种基于法向量的深度图像缺陷检测算法,主要思路是利用法向量的余弦相似度来描述法向量的变化,根据输入的原始深度图像计算其每个像素点的法向量,然后计算每个像素点与其邻域的法向量变化度量获得一张法向量梯度图像,接着根据法向量梯度图像使用双阈值分割获得候选的缺陷区域,最后验证候选区域的深度偏差是否达到了缺陷的定义,将满足缺陷定义的候选区域保留下来作为最终结果;

    4)根据真实的钢板数据生成了一个人工合成的深度图像数据集。生成的过程主要是先根据真实钢板图像获得钢板模型,然后根据真实的缺陷信息获得缺陷模型,最后将缺陷模型与钢板模型结合起来合成带缺陷的深度图像。在生成的过程中会对缺陷模型随机的选择旋转角度以及缩放尺度进行旋转和缩放来尽可能的近似真实的深度数据。


    \section{论文的组织结构}
    \esection{Thesis Structure}
    本篇论文一共六个章节,各个章节的内容安排如下:

    第一章:绪论。主要介绍了研究热态连铸坯实时在线缺陷检测系统的背景与意义,实现热态连铸坯在线缺陷检测技术可以有效地避免带缺陷的连铸坯毫无意义的继续深加工,而无缺陷的连铸坯能够直接送入下一步热轧工艺。接着,介绍了目前该领域国类外的研究现状,最后说明了本论文的组织结构安排;

    第二章:连铸坯表面缺陷检测方法概述。主要是调研整理并介绍了近年来国内外在热态连铸坯表面缺陷检测领域的各种方法,主要分为传统无损检测方法、基于RGB图像的方法和基于深度图像的方法。接着,对这三种方法的典型计算过程进行了简要阐述,并对方法的优缺点进行了比较和分析;

    第三章:基于显著性图的缺陷检测算法。主要是详细介绍了基于显著性图方法的流程和具体操作,主要思路是首先是用改进的显著性检测来获取图像的显著性图并去除图像中的黑斑,然后使用canny边缘检测提取边缘图像,最终使用形态学操作对边缘图像进行处理获得最终的缺陷区域。最终我们收集了一些生产线现场的钢板图像并进行了实验和分析;

    第四章:基于基准面拟合的缺陷检测算法。主要是详细介绍了基于基准面拟合方法的基本流程和具体操作,主要思路是根据输入的深度图像拟合出其基准面,再根据拟合的基准面以及原始深度图像获得缺陷种子点,最后在原始深度图像上使用基于种子点的图像分割算法获取最终的缺陷轮廓。接着,我们使用人工标注和人工合成的手段生成了一个钢板深度图像数据库,并在这个数据库上进行了实验,最终分析了实验结果;

    第五章:基于法向量的缺陷检测算法。主要是详细介绍了基于法向量方法的基本流程和具体操作,主要思路是利用法向量的余弦相似度来描述法向量的变化,根据输入的原始深度图像计算其每个像素点的法向量,然后计算每个像素点与其邻域的法向量变化度量获得一张法向量梯度图像,接着根据法向量梯度图像使用双阈值分割获得候选的缺陷区域,最后验证候选区域的深度偏差是否达到了缺陷的定义,将满足缺陷定义的候选区域保留下来作为最终结果。最终我们在人工合成数据集上对进行了实验和分析;

    第六章:结论与展望。主要是对本论文进行了总结,分析了本文中提出的各个方法的优点与不足,最终我们对不足之处进行了分析和思考并提出了未来的工作和研究方向。

