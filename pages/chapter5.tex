% multiple1902 <multiple1902@gmail.com>
% intro.tex
% Copyright 2011~2012, multiple1902 (Weisi Dai)
% https://code.google.com/p/xjtuthesis/
%
% It is strongly recommended that you read documentations located at
%   http://code.google.com/p/xjtuthesis/wiki/Landing?tm=6
% in advance of your compilation if you have not read them before.
%
% This work may be distributed and/or modified under the
% conditions of the LaTeX Project Public License, either version 1.3
% of this license or (at your option) any later version.
% The latest version of this license is in
%   http://www.latex-project.org/lppl.txt
% and version 1.3 or later is part of all distributions of LaTeX
% version 2005/12/01 or later.
%
% This work has the LPPL maintenance status `maintained'.
%
% The Current Maintainer of this work is Weisi Dai.
%

\chapter{基于法向量的缺陷检测算法}
\echapter{Normal Vector based Defect Detection Method}

    \section{方法概述}
    \esection{Outline}
    

    

    \section{计算法向量梯度图}
    \esection{Gradient Map of Normal Vector}
   

    \section{双阈值分割}
    \esection{Dual Threshold Segmentation}
   
   

    \section{深度验证}
    \esection{Verification of Depth}
   
    \section{实验结果}
    \esection{Experimental Results}
        \subsection{数据集与评价方法}
        \esubsection{Data-set and Evaluation Protocol}
        
        \subsection{实验结果与分析}
        \esubsection{Experimental Results and Analysis}
        


    \section{本章小结}
    \esection{Brief Summary}
   

