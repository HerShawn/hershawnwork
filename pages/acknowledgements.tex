% multiple1902 <multiple1902@gmail.com>
% acknowledgements.tex
% Copyright 2011~2012, multiple1902 (Weisi Dai)
% https://code.google.com/p/xjtuthesis/
%
% It is strongly recommended that you read documentations located at
%   http://code.google.com/p/xjtuthesis/wiki/Landing?tm=6
% in advance of your compilation if you have not read them before.
%
% This work may be distributed and/or modified under the
% conditions of the LaTeX Project Public License, either version 1.3
% of this license or (at your option) any later version.
% The latest version of this license is in
%   http://www.latex-project.org/lppl.txt
% and version 1.3 or later is part of all distributions of LaTeX
% version 2005/12/01 or later.
%
% This work has the LPPL maintenance status `maintained'.
%
% The Current Maintainer of this work is Weisi Dai.
%
不知不觉研究生生涯就快结束了,回首研究生这三年,不敢说收获颇丰,也不能说碌碌无为,但无论我学到了什么,无论我有怎样的成长,都离不开父母、老师还有各位同学的关心和帮助,在这里我要对各位说一声谢谢。

感谢课题组的宋永红老师和张元林老师,是两位老师严格的要求让我在实验室有所进步。还记得我刚进实验室的时候,对图像处理方面一窍不通,是宋老师给我指明了研究方向,教会我研读文献,帮助我科研入门。张老师则经常强调总结和计划的重要性,让我明白凡事预则立不预则废。感谢两位老师一直以来的帮助和支持。

感谢实验室的各位同仁,王晓冰师兄、张云师兄、郁冲师兄、陈晓师兄、高丹妮师姐、龚晨师兄、李伟师兄、梁朝寓师兄、冯媛媛师姐、尚玉飞、刘衍峰、段露、张植、魏盛华、贺翔、赵路、戴觊婧、吴晨、杜鹏、任泽宇、赵君、李晓玉、王欣杨、杨柳,感谢各位同学的帮助。每一次开小会一起头脑风暴探讨没读懂的论文中的各种难点,每一次研究项目中各种问题的解决方案,每一次休息时吐槽生活学习中的各种事情,都成了过去时光里的种种,镶嵌在时间长河的相框里,这些种种弥足珍贵,未来不可期,往事不可追。祝各位好运,祝各位开心。

感谢我的舍友黄山、王铮和殷浩天,宿舍的集体生活给我们奠定了深厚的友谊,也让我看到了他们每个人身上的闪光点,感谢他们在潜移默化中教会我许多道理,让我受益良多。

感谢我的专业同学,他们提供了很多有价值的建议,感谢他们的许多支持和帮助。

然后感谢我的父母,在任何我要感谢谁的时候,他们都不会被漏掉。

最后,感谢在百忙之中抽出时间为我评阅论文的诸位老师。由于我的学术水平有限,所写论文难免有不足之处,恳请各位老师和学友批评指正。

