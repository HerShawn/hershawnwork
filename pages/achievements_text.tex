% multiple1902 <multiple1902@gmail.com>
% acknowledgements.tex
% Copyright 2011~2012, multiple1902 (Weisi Dai)
% https://code.google.com/p/xjtuthesis/
%
% It is strongly recommended that you read documentations located at
%   http://code.google.com/p/xjtuthesis/wiki/Landing?tm=6
% in advance of your compilation if you have not read them before.
%
% This work may be distributed and/or modified under the
% conditions of the LaTeX Project Public License, either version 1.3
% of this license or (at your option) any later version.
% The latest version of this license is in
%   http://www.latex-project.org/lppl.txt
% and version 1.3 or later is part of all distributions of LaTeX
% version 2005/12/01 or later.
%
% This work has the LPPL maintenance status `maintained'.
%
% The Current Maintainer of this work is Weisi Dai.
%
【论文与专利】

[1]宋永红,贺翔,张元林.一种基于二叉树的文本行精确定位方法[P].中国发明专利,申请号:2016108504496

[2]宋永红,贺翔,张元林.一种用以增强文字与背景差异的边缘响应统计变换方法[P].中国发明专利,申请号:2016108503972

[3]He Xiang, Song Yonghong, Zhang Yuanlin. Scene Text Detection based on Statistic Edge Response and MSER, 25th National Conference on Multimedia Technology(NCMT),2016.

[4]He Xiang, Song Yonghong, Zhang Yuanlin. Scene Text Detection based on Skeleton-cut Detector, 24th International Conference on Image Processing(ICIP),2017.

[5]He Xiang, Song Yonghong, Zhang Yuanlin. A coarse-to-fine scene text detection method based on Skeleton-cut detector and Binary-Tree-Search based rectification, Pattern Recognition Letters, 2018已投稿



【参与项目】

[1]海尔智能冰箱中的容器包装文字识别项目(2017.6 - 2017.12) 项目组长

本项目目标是在低分辨率冰箱摄像头拍摄的图片中,检测并识别出食品包装上的有用信息。首先利用基于深度网络的目标检测方法来定位文字,然后采用CRNN模型对定位到的文字进行识别,此外引入基于ReID的图像匹配算法来辅助提升识别效果。本人负责完成的工作有:制订工作计划、调研相关工作、仿真算法、制作数据集、实现算法、设计和搭建系统平台、做实验测试结果、调试系统以及在每个阶段撰写技术报告。



%\clearpage


