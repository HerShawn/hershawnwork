% multiple1902 <multiple1902@gmail.com>
% conclusion.tex
% Copyright 2011~2012, multiple1902 (Weisi Dai)
% https://code.google.com/p/xjtuthesis/
%
% It is strongly recommended that you read documentations located at
%   http://code.google.com/p/xjtuthesis/wiki/Landing?tm=6
% in advance of your compilation if you have not read them before.
%
% This work may be distributed and/or modified under the
% conditions of the LaTeX Project Public License, either version 1.3
% of this license or (at your option) any later version.
% The latest version of this license is in
%   http://www.latex-project.org/lppl.txt
% and version 1.3 or later is part of all distributions of LaTeX
% version 2005/12/01 or later.
%
% This work has the LPPL maintenance status `maintained'.
%
% The Current Maintainer of this work is Weisi Dai.
%
\chapter{结论与展望}
\echapter{Conclusions and Future Work}

    \section{结论}
    \esection{Conclusions}
    
    自然场景图像中的文字检测是模式识别和机器视觉领域的热点研究方向,其可以在多种人工智能系统中得到应用:如无人驾驶、导盲、内容检索以及工业自动化。但是因为存在背景复杂、文字多样以及图像退化等问题的影响,导致在场景图像中精确的检测文字是一件很有挑战性的研究课题。本文对自然场景图像中的文字检测的相关研究工作进行调研和分析,并提出两个新颖的场景文字检测子。主要工作如下所示:
    
    首先对自然场景图像中的文字检测领域内的研究现状及其问题进行了调研和分析,将文字检测方法分成两大类主流方法:基于区域的方法和基于连通部件的方法。然后各举一些经典方法以分析优缺点,从而奠定本文研究工作的理论基础。
    
    然后提出了一种基于边缘骨架切割的文字检测方法用以解决边缘粘连问题。首先利用结构化边缘检测方法得到边缘响应图。接着在每个二值边缘图上,通过细化操作得到其边缘骨架图,并通过8领域内像素点分析所找到的边缘骨架结点。断开粘连点得到候选的文字边缘骨架,然后经过形态学滤除来过滤掉大部分明显不是文字的边缘骨架,剩余的非文字边缘骨架通过CNN分类器来滤除。最后基于非极大值抑制和文本行聚集得到定位结果。然而文字包围盒可能存在重叠率低的缺陷而导致文字定位不够精准。
    
    接着提出一种基于二叉树搜索的文字检测方法用于解决上一个方法中文字包围盒定位不准的缺陷。首先对候选文字包围盒计算统计边缘响应,接着通过水平投影、求取梯度以及执行非极大值抑制等操作获得候选文本行。最后由生成的候选文本行中建立起二叉树型的搜索空间,并根据优化策略从搜索空间中找到最优的文本行定位结果。该结果即是经过优化后得到的文字细致定位结果,相比于边缘骨架切割文字检测子的粗略定位结果而言检测精度得到了大幅提升。

    \section{展望}
    \esection{Future Work}

